\documentclass[10pt,a4paper]{article}
\usepackage[utf8]{inputenc}
\usepackage{amsmath}
\usepackage{amsfonts}
\usepackage{amssymb}
\usepackage{graphicx}
\usepackage{caption}
\usepackage{subcaption}
\usepackage{float}
\usepackage{minted}
\usepackage{cite}
\usepackage[top=1.0in, bottom=1.0in, left=0.75in, right=0.75in]{geometry}

\usepackage{url}

\author{Chittaranjan Srinivas Swaminathan and Anders Wikstrom}
\title{Scortec ER-I Manipulator Control using Arduino Due}
\begin{document}
\maketitle
\tableofcontents
\newpage
\section{Introduction}

\section{Description of task}
The Scortec ER-I is a 5-DOF manipulator from Eshed Robotec whose
control box is missing. The goal of this project is to use the Arduino
Due board and a motor shield to make a new controller for the
manipulator. 

Figure \ref{fig:axes} shows a sketch of the manipulator and the axes.

\begin{figure}[h]
    \centering
    \includegraphics{axes.png}
    \caption{Scortec ER-I}
    \label{fig:axes}
\end{figure}

The task of creating a controller for the manipulator involves the
following steps:
\begin{enumerate}
\item Reverse engineer the motor and encoder connections and make a
  pinout diagram for the D50 connector on the manipulator.
\item Program the microcontroller to read the encoders.
\item Write control routines for position and velocity control.
\item Extend the control routines to more than one joint.
\end{enumerate}

The following sections describe each of these tasks and challenges
faced while executing them.

\section{Reverse Engineering}

In this task, we set out to map the pins on the D50 connector to the
individual motors and encoders. Once the pin-out diagram was in place
we could connect the Arduino and read the encoder pins. However, we
faced some problems while reading encoder outputs. The following
sections describe each of these tasks in detail. \\

\subsection{Pinout}
The primary task was to figure out the what each pin on the D50
connector meant. For this task we first took one motor and used the
multimeter to test for continuity. The process was repeated for each
motor and encoder board. Figuring out the circuit on the encoder board
was also important so as to understand which pins were the power
supply pins and which pins were the output pins.\\

The figure \ref{fig:dsub} is the pinout diagram. Note
that: \begin{itemize} 
\item \( E^a_b \) refers to output of encoder \textit{b} on joint \textit{a}.
\item \( GND^a \) refers to ground for encoder board on joint \textit{a}.
\item \( V^a_{cc} \) refers to power supply for encoder board on joint
\textit{a}.
\item \( M^a+ \) refers to the positive pin on motor \textit{a}.
\item \( M^a- \) refers to the negative pin on motor \textit{a}.
\end{itemize}


\begin{figure}[h]
    \centering
    \includegraphics[scale=0.3]{dsub50.png}
    \caption{Pinout diagram for the D50 connector.}
    \label{fig:dsub}
\end{figure}

The motor has two pins (one positive and one negative) on the
connector. Each encoder board has a pair of encoders, thus enabling us
to read the incremental position of the joint as well as the direction
in which the joint is moving. The board has 6 pins but only 4 had
corresponding pins on the D50 connector. Also, it is important to
figure out what each of these pins meant.\\

To do this, we used the following steps:
\begin{enumerate}
\item The basic circuit for a Photodiode-LED pair is shown in figure
  \ref{fig:photodiodeLEDPair}. 
\item Detach the encoder board and use continuity test to figure out
  which components are connected.
\item Using the above two, figure out the circuit diagram of the
  board. This was found to be equivalent to the circuit in figure
  \ref{fig:encoderCircuit}. 
\item Finally, relate each pin on the encoder board to the
  circuit. 
\end{enumerate}

\begin{figure}[h]
    \centering
    \includegraphics[scale=0.5]{SimpleEncoder.jpg}
    \caption{Basic circuit for Photodiode-LED pair.}
    \label{fig:photodiodeLEDPair}
\end{figure}

\begin{figure}[h]
    \centering
    \includegraphics[scale=0.5]{EncoderCircuit.jpg}
    \caption{Equivalent Encoder circuit with two Photodiode-LED
      pairs. The photodiode is circled.}
    \label{fig:encoderCircuit}
\end{figure}

At this point, we were ready to read the encoder inputs from the
microcontroller. 

\subsection{Debouncing}

Routines to read the encoders were implemented from
\cite{ArduinoPlaygroundRE}.
The first test was to move the motor shaft by hand and check if the
rotation produced the necessary encoder ticks. This step gave positive
results. The next step was to move the joint by hand and see if the
incremental position was stable between motions. This step also gave
positive results. The position was only off by a few ticks everytime
the joint was moved from one extremum position to another. \\

However, when the joint was moved using PWM input, we obtained far
more encoder ticks than when moved manually. Furthur inquiry lead us
to use a debouncer circuit shown in figure \ref{fig:rcfilter}. This is
also take from the Arduino Playground page on rotary encoders
\cite{ArduinoPlaygroundRE}.\\

\begin{figure}[h]
    \centering
    \includegraphics[scale=0.5]{debouncer.jpg}
    \caption{An RC filter for debouncing.}
    \label{fig:rcfilter}
\end{figure}

After introducing the RC Filter (hardware debounce circuit), the
encoder readings seemed to be fairly stable. Minor error were still
observed. 

\subsection{Current sensing}
We use current sensing to detect when we reach an end position where the arm is unable to move further. 
PWM modulation means varying current.

Moving average?

lowpass filter?

%Do we currently have a use for the filtering?

\section{Control routines}

In this section, the control routines and organization of code is
described. We employ a PID controller for velocity 

\subsection{Calibration}

Each joint is calibrated as follows:
\begin{itemize}
\item The joint to be calibrated is first moved in the negative
  direction\footnote{Negative direction is the direction that results
    in negative encoder readings} till the extremum is reached. The
  extremum point is detected by measuring the current.
\item At this point the encoder value is reset. 
\item The joint is then moved to the other extremum in the positive
  direction, at which point calibration ends.
\item The encoder value at positive extremum is the total encoder
  ticks for full range movement of the joint.
\end{itemize}
%%%%
%%
%% Write something about why this is done and how we do it in code.
%%
%%%%

\subsubsection{Results from calibration}
The range of the joints was taken from the datasheet of the
manipulator. If we calculate the total encoder ticks for the full
range of a joint, then we can compute a ratio to convert angular
displacement/velocity described in encoder ticks to
degrees. We have,
\[ \alpha = \frac{range\ of\ joint}{measured\ encoder\ ticks}\]

where \(\alpha\) is the ratio described above\footnote{The ratio
  \(\alpha\) is simply called 'ratio' in code.}. The unit of
\(\alpha\) is \( \frac{encoder\ ticks}{degree}\). \\

We observed that the number of ticks over the full range varied more
than a few tens of ticks. Hence, we averaged the measured ticks and
computed a ratio from the average. \\ 

\begin{tabular}{ | l | r | r |}
\hline
- & \textbf{Joint 1} & \textbf{Joint 3} \\
\hline
1 & 9748 & 3195 \\
\hline
2 & 9924 & 3178 \\
\hline
3 & 9832 & 3181 \\
\hline
4 & 9850 & 3193 \\
\hline
5 & 9796 & 3194 \\
\hline
\textbf{Mean} & 9830 & 3188.2 \\
\hline
\textbf{Ratio} & 30.71875 & 17.049 \\
\hline

\end{tabular}

\subsection{Velocity Control}
The velocity of the joints are controlled by a PID controller. 

\subsection{Position Control}
The position controller looks at the direction the arm have to move to reach the desired position. When is given a new position 

\subsection{Overall organization of code}
Each joint is controlled by an independent motor controller that contains all information needed to control its joint. The motor controller stores things such as PIN numbers for encoders, current sensing and stores state information used for the encoders and PID controllers.


\section{Evaluation of the system}

Two methods:
\begin{itemize}
\item Accuracy and precision of the incremental optical encoders.
\item Accuracy and precision of the control system.
\end{itemize}

\subsection{Accuracy and Precision for Joint 1}
How did we measure?
What did we get?

\section{Future Work}
\begin{itemize}
\item After obtaining ratios for each joint from calibration, we could
  replace the calibration routines with ``home'' routine. When the arm
  is HOME-d, each joint simply moves to the negative extremum
  position. 
\item Add all joints. Framework is ready.
\item Do some forward kinematics and output tf to ROS.
\item Write a controller for Moveit.
\end{itemize}

\bibliographystyle{ieeetr}
\bibliography{report}

\end{document}